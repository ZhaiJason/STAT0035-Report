\appendix

\chapter{Appendix}

\section{Package Source Code} \label{apd:source_code}

This appendix includes \textit{selected} key components of the package functions. Notice that for code readability in this PDF file, some functions and function comments are omitted. Therefore, visiting the package's \hyperref{https://github.com/ZhaiJason/rmcop}{site}{rmcop Repository}{GitHub Repository} is highly recommended. The online repository stores the full directory of the package with detailed installation instructions. This, however, comes with the risk of breaking anonymity, as some parts of the GitHub site may reveal the creator's information.

\newpage

\section{Option.R}

\begin{Rminted}
# Creating new option object ===================================================
#' Initialise new `option` class object
#' Create an S3 class object which characterised an option of interest. The returned object is then used together with a `"env"` class object together as inputs of the `price.option` function for further option pricing.
option <- function(style, option = "vanilla", type, K, t, ...) {

    obj <- list(
        "style" = style,
        "type" = type,
        "K" = K,
        "t" = t
    )

    # Distribute to corresponding initialisation function according to specified arguments
    if (option == "vanilla") {
        obj <- option.vanilla(obj)
    } else {
        obj <- option.exotic(obj, option, ...)
    }
    obj
}

## Vanilla =====================================================================
#' Initialise vanilla option
option.vanilla <- function(obj) {
    class(obj) <- c("vanilla", "option")
    obj
}

## Exotic ======================================================================
#' Initialise exotic option
option.exotic <- function(obj, option, ...) {
    obj <- get(paste0("option.exotic.", option))(obj, ...) # Direct to corresponding initialisation function
    obj
}

#' Initialise Asian option
option.exotic.asian <- function(obj, is.avg_price = TRUE) {
    obj <- c(
        obj,
        "is.avg_price" = is.avg_price
    )
    class(obj) <- c("asian", "option")
    obj
}

#' Initialise barrier option
option.exotic.barrier <- function(obj, barrier, is.knockout = TRUE) {
    obj <- c(
        obj,
        "barrier" = barrier,
        "is.knockout" = is.knockout
    )
    class(obj) <- c("barrier", "option")
    obj
}

#' Initialise binary option
option.exotic.binary <- function(obj, payout) {
    obj <- c(
        obj,
        "payout" = payout
    )
    class(obj) <- c("binary", "option")
    obj
}

#' Initialise lookback option
option.exotic.lookback <- function(obj, is.fixed = TRUE) {
    obj <- c(
        obj,
        "is.fixed" = is.fixed
    )
    class(obj) <- c("lookback", "option")
    obj
}

# Creating new option environment ==============================================

#' Initialise new `env` class object
#' Create an S3 class object which characterised the market environment used when pricing the option of interest. The returned object is used together with an `"option"` class object as inputs of the pricing function `price.option`.
option.env <- function(method = "mc", S, r, q = 0, sigma, n = NULL, steps = NULL) {
    env <- list(
        "method" = method,
        "S" = S,
        "r" = r,
        "q" = q,
        "sigma" = sigma,
        "n" = n,
        "steps" = steps
    )
    class(env) <- "env"
    env
}
\end{Rminted}

\newpage

\section{Price.R}

\begin{Rminted}
# Top Level Pricing Functions for Options ======================================

#' Pricing `option` Class Object
#' The all-in-one pricing function for `rmcop`. User should predefine an `"option"` class object and an `"env"` class object as inputs of `price.option`. According to the pricing method selected, the user shall provide additional inputs as instructions for the pricing function.
price.option <- function(obj, env, method = env$method, n = env$n, ... , all = FALSE) {
    check.method(method)
    res <- get(paste0("price.option.", method))(obj = obj, env = env, n = n, ..., all = all)
    res
}

#' Option Pricing via Monte Carlo Method
price.option.mc <- function(obj, env, n, steps = env$steps, all, ...) {
    if (is.null(n)) stop("n is not specified")
    if (is.null(steps)) {
        steps <- 1
        warning("steps is not specified, evaluated to default value 1")
    }
    res <- get(paste0(class(obj)[1], ".mc"))(obj = obj, env = env, n = n, steps = steps, all = all, ...)
    res
}

#' Option Pricing via Black-Scholes Model
price.option.bs <- function(obj, env, n, all) {
    if (obj$style == "american") {
        if (env$q != 0 | obj$type != "call") {
            warning("Black Scholes valuation should only apply on European european options or American call option with non-dividend paying asset, the result here will be based on European options pricing and may not be accurate.")
        }
    }
    res <- get(paste0(class(obj)[1], ".bs"))(obj, env, all)
    res
}

#' Option Pricing via Binomial Lattice Model
price.option.binomial <- function(obj, env, n, u, d, all) {
    res <- get(paste0(class(obj)[1], ".binomial"))(obj = obj, env = env, n = n, u = u, d = d, all = all)
    res
}
\end{Rminted}

\newpage

\section{MonteCarlo.R}

\begin{Rminted}
#' Engine for Monte Carlo Simulation on Asset Price Trajectory
mc.engine <- function(type, K, t, S0, r, q, sigma, n, steps) {

    dt <- t / steps

    # Generate n random samples from N(0,1)
    Z <- matrix(rnorm(n * steps), ncol = n)

    # Get logarithm of price change per step
    increment <- (r - q - sigma^2 / 2) * dt + sigma * sqrt(dt) * Z

    # Use vectorised MC method to compute logarithm of S(t) at each time step
    logS <- log(S0) + apply(increment, 2, cumsum)
    S <- exp(logS) # Obtain the simulated stock price by taking exponential
    S <- S * exp(-q * t) # Adjust the asset price for dividends
    S <- rbind(S0, S) # Add column of current price
    S
}

#' Compute Monte Carlo Model's Standard Error
mc.SE <- function(C_i, C.mean, n) {
    s_c <- sqrt(sum(C_i - C.mean)^2 / (n - 1))
    SE <- s_c / sqrt(n)
    SE
}

# Model ========================================================================

#' Pricing vanilla option using Monte Carlo method
vanilla.mc <- function(obj, env, n, steps, all = FALSE, plot = FALSE, ...) {

    # Check object integrity, CAN BE OMITTED
    check.args(obj, env, option = "vanilla")

    # Extract objects properties
    style <- obj$style
    type <- obj$type
    K <- obj$K
    t <- obj$t
    S0 <- env$S
    r <- env$r
    q <- env$q
    sigma <- env$sigma

    # Apply MC
    if (style == "european") {
        S <- mc.engine(type, K, t, S0, r, q, sigma, n, steps)
    } else if (style == "american") {
        stop("American option pricing via MC not supported")
    }

    # Compute corresponding discounted price at current time
    # Vanilla option only consider option price at maturity, so only take the last column of the price matrix S_i
    if (type == "call") {
        C <- exp(-r * t) * pmax(S[steps + 1, ] - K, 0)
    } else if (type == "put") {
        C <- exp(-r * t) * pmax(K - S[steps + 1, ], 0)
    }

    # Obtain the estimate of option price by taking average of C
    C.mean <- mean(C, ...)

    # Compute standard error of the estimate
    SE <- mc.SE(C, C.mean, n)

    # Plot function
    if (plot) {mc.plot(S, ...)}

    # Return result
    if (all) {
        # Return actual object environment used
        env <- option.env(method = "mc", S = S, r = r, q = q, sigma = sigma, n = n, steps = steps)
        list(
            "env" = env,
            "price" = C.mean,
            "SE" = SE,
            "S" = S,
            "C" = C
        )
    } else {
        C.mean
    }
}

#' Pricing Asian option using Monte Carlo method
asian.mc <- function(obj, env, n, steps, all = FALSE, plot = FALSE, ...) {
    # Check object integrity, CAN BE OMITTED
    check.args(obj, env, option = "asian")

    # Extract objects properties
    style <- obj$style
    type <- obj$type
    K <- obj$K
    t <- obj$t
    is.avg_price <- obj$is.avg_price
    S0 <- env$S
    r <- env$r
    q <- env$q
    sigma <- env$sigma

    # Apply MC
    if (style == "european") {
        S <- mc.engine(type, K, t, S0, r, q, sigma, n, steps)
    } else if (style == "american") {
        stop("American option pricing via MC not supported")
    }

    # Compute the mean of each replication's price movement, as required to compute Asian option pay-off
    S.mean <- apply(S, 2, mean)

    # Compute corresponding discounted price at current time
    if (is.avg_price) { # Compute payoff for average price Asian option
        if (type == "call") {
            C <- exp(-r * t) * pmax(S.mean - K, 0)
        } else if (type == "put") {
            C <- exp(-r * t) * pmax(K - S.mean, 0)
        }
    } else { # Compute payoff for average strike Asian option
        if (type == "call") {
            C <- exp(-r * t) * pmax(S[steps + 1, ] - S.mean, 0)
        } else if (type == "put") {
            C <- exp(-r * t) * pmax(S.mean - S[steps + 1, ], 0)
        }
    }

    # Obtain the estimate of option price by taking average of C
    C.mean <- mean(C)

    # Compute standard error of the estimate
    SE <- mc.SE(C, C.mean, n)

    # Plot function
    if (plot) {mc.plot(S, ...)}

    # Return result
    if (all) {
        # Return actual object environment used
        env <- option.env(method = "mc", S = S, r = r, q = q, sigma = sigma, n = n, steps = steps)
        list(
            "env" = env,
            "price" = C.mean,
            "SE" = SE,
            "S" = S,
            "C" = C
        )
    } else {
        C.mean
    }
}

#' Pricing Barrier option using Monte Carlo method
barrier.mc <- function(obj, env, n, steps, all = FALSE, plot = FALSE, ...) {
    # Check object integrity, CAN BE OMITTED
    check.args(obj, env, option = "barrier")

    # Extract objects properties
    style <- obj$style
    type <- obj$type
    K <- obj$K
    t <- obj$t
    barrier <- obj$barrier
    is.knockout <- obj$is.knockout
    S0 <- env$S
    r <- env$r
    q <- env$q
    sigma <- env$sigma

    # Apply MC
    if (style == "european") {
        S <- mc.engine(type, K, t, S0, r, q, sigma, n, steps)
    } else if (style == "american") {
        stop("American option pricing via MC not supported")
    }

    # Check if price movements corssed the specified barrier
    S.min <- apply(S, 2, min)
    S.max <- apply(S, 2, max)
    is.cross <- (S.min < barrier & S.max > barrier)

    # Compute corresponding discounted price at current time
    if (type == "call") {
        C <- exp(-r * t) * pmax(S[steps + 1, ] - K, 0)
    } else if (type == "put") {
        C <- exp(-r * t) * pmax(K - S[steps + 1, ], 0)
    }

    # Adjust for knock-out & knock-in situations for a barrier option
    if (is.knockout) {
        C <- ifelse(is.cross, 0, C)
    } else {
        C <- ifelse(is.cross, C, 0)
    }

    # Obtain the estimate of option price by taking average of C
    C.mean <- mean(C)

    # Compute standard error of the estimate
    SE <- mc.SE(C, C.mean, n)

    # Plot function
    if (plot) {mc.plot(S, ...)}

    # Return result
    if (all) {
        # Return actual object environment used
        env <- option.env(method = "mc", S = S, r = r, q = q, sigma = sigma, n = n, steps = steps)
        list(
            "env" = env,
            "price" = C.mean,
            "SE" = SE,
            "S" = S,
            "C" = C
        )
    } else {
        C.mean
    }
}

#' Pricing Barrier option using Monte Carlo method
binary.mc <- function(obj, env, n, steps, all = FALSE, plot = FALSE, ...) {

    # Check object integrity, CAN BE OMITTED
    check.args(obj, env, option = "binary")

    # Extract objects properties
    style <- obj$style
    type <- obj$type
    K <- obj$K
    t <- obj$t
    payout <- obj$payout
    S0 <- env$S
    r <- env$r
    q <- env$q
    sigma <- env$sigma

    # Apply MC
    if (style == "european") {
        S <- mc.engine(type, K, t, S0, r, q, sigma, n, steps)
    } else if (style == "american") {
        stop("American option pricing via MC not supported")
    }

    # Compute corresponding discounted price at current time
    # Compute uniform payout for each MC replication of a binary option
    C <- matrix(exp(-r * t) * payout, nrow = n, ncol = 1)
    if (type == "call") {
        C <- ifelse(S[steps + 1, ] > K, C, 0)
    } else if (type == "put") {
        C <- ifelse(S[steps + 1, ] < K, C, 0)
    }

    # Obtain the estimate of option price by taking average of C
    C.mean <- mean(C)

    # Compute standard error of the estimate
    SE <- mc.SE(C, C.mean, n)

    # Plot function
    if (plot) {mc.plot(S, ...)}

    # Return result
    if (all) {
        # Return actual object environment used
        env <- option.env(method = "mc", S = S, r = r, q = q, sigma = sigma, n = n, steps = steps)
        list(
            "env" = env,
            "price" = C.mean,
            "SE" = SE,
            "S" = S,
            "C" = C
        )
    } else {
        C.mean
    }
}

#' Pricing Lookback option using Monte Carlo method
lookback.mc <- function(obj, env, n, steps, all = FALSE, plot = FALSE, ...) {
    # Check object integrity, CAN BE OMITTED
    check.args(obj, env, option = "lookback")

    # Extract objects properties
    style <- obj$style
    type <- obj$type
    K <- obj$K
    t <- obj$t
    is.fixed <- obj$is.fixed
    S0 <- env$S
    r <- env$r
    q <- env$q
    sigma <- env$sigma

    # Apply MC
    if (style == "european") {
        S <- mc.engine(type, K, t, S0, r, q, sigma, n, steps)
    } else if (style == "american") {
        stop("American option pricing via MC not supported")
    }

    # Check if price movements crossed the specified barrier
    S.min <- apply(S, 2, min)
    S.max <- apply(S, 2, max)

    # Compute corresponding discounted price at current time
    if (type == "call") {
        if (is.fixed) {
            C <- exp(-r * t) * pmax(S.max - K, 0)
        } else {
            C <- exp(-r * t) * (S[steps + 1, ] - S.min)
        }
    } else if (type == "put") {
        if (is.fixed) {
            C <- exp(-r * t) * pmax(K - S.min, 0)
        } else {
            C <- exp(-r * t) * (S.max - S[steps + 1, ])
        }
    }

    # Obtain the estimate of option price by taking average of C
    C.mean <- mean(C)

    # Compute standard error of the estimate
    SE <- mc.SE(C, C.mean, n)

    # Plot function
    if (plot) {mc.plot(S, ...)}

    # Return result
    if (all) {
        # Return actual object environment used
        obj <- option(style = style, option = "lookback", type = type, K = K, t = t, is.fixed = is.fixed)
        env <- option.env(method = "mc", S = S, r = r, q = q, sigma = sigma, n = n, steps = steps)
        list(
            "env" = env,
            "C.price" = C.mean,
            "SE" = SE,
            "S" = S,
            "C" = C
        )
    } else {
        C.mean
    }
}
\end{Rminted}

\newpage

\section{BlackScholes.R}

\begin{Rminted}
#' Compute d1 in Black-Scholes Formula
bs.d1 <- function(S, K, t, r, q, sigma) {
    (log(S / K) + (r - q + sigma^2 / 2) * t) / (sigma * sqrt(t))
}

# Model ========================================================================

#' Pricing vanilla option using Black-Scholes model
vanilla.bs <- function(obj, env, all = FALSE) {
    type <- obj$type
    K <- obj$K
    t <- obj$t
    S <- env$S
    r <- env$r
    q <- env$q
    sigma <- env$sigma

    d1 <- bs.d1(S, K, t, r, q, sigma)
    d2 <- d1 - sigma * sqrt(t)

    if (type == "call") {
        price <- exp(-q * t) * S * pnorm(d1) - exp(-r * t) * K * pnorm(d2)
    } else if (type == "put") {
        price <- exp(-r * t) * K * pnorm(-d2) - exp(-q * t) * S * pnorm(-d1)
    }

    if (all) {
        list(
            obj,
            env,
            "price" = price
        )
    } else {
        price
    }
}
\end{Rminted}

\newpage

\section{Binomial.R}

\begin{Rminted}
# Engine =======================================================================
#' Generating Binomial Tree Matrix
Binomial.tree <- function(S0, u, d, n) {
    if (is.null(n)) stop("Number of binomial tree steps n is not defined.")
    dim <- n + 1
    S <- matrix(nrow = dim, ncol = dim) # Predefine the size
    temp <- c(S0)
    for (i in 1:dim) {
        S[i, ] <- c(temp, rep(NA, dim - i))
        temp <- c(temp[1] * d, temp * u)
    }
    S
}

#' Backward Recursive Risk-neutral Valuation for Binomial Model
Binomial.rnv <- function(fn, p, q, r, n, dt, K = NULL, S = NULL,
                         type = "call",
                         style = "european") {
    if (style == "european") {
        rnv.european <- function(fn, p, r, dt, state = n) {
            if (state <= 0) {
                fn
            } else {
                l <- state # Specify the length l of the vector containing prices of next state looking back by 1 step
                fm <- sapply(1:l, function(i) {exp(-r * dt) * (p * fn[i + 1] + q * fn[i])})
                rnv.european(fm, p, r, dt, state = state - 1)
            }
        }
        rnv.european(fn, p, r, dt)
    } else if (style == "american") {
        rnv.american <- function(fn, p, r, dt, K, S, type = type, state = n) {
            if (state <= 0) {
                fn
            } else {
                l <- state
                fm <- sapply(1:l, function(i) {
                    if (type == "call") {
                        max(max(S[state, i] - K, 0), exp(-r * dt) * (p * fn[i + 1] + q * fn[i]))
                    } else if (type == "put") {
                        max(max(K - S[state, i], 0), exp(-r * dt) * (p * fn[i + 1] + q * fn[i]))
                    }
                })
                rnv.american(fm, p, r, dt, K, S, type = type, state = state - 1)
            }
        }
        rnv.american(fn, p, r, dt, K, S, type = type)
    } else {
        stop("Invalid option style. `style` must be specified to either 'european' or 'american'.")
    }
}

# Model ========================================================================

#' Binomial model for pricing vanilla options
vanilla.binomial <- function(obj, env, n, u, d, all = FALSE) {

    style = obj$style
    type = obj$type
    K = obj$K
    t = obj$t
    S = env$S
    r = env$r

    dt <- t / n # Compute delta t
    p <- (exp(r * dt) - d) / (u - d) # Compute riskless probability for an upward price movement
    q <- 1 - p # Compute riskless probability for a downward price movement

    # Store the stock prices of the tree for each state in a matrix `price_tree`
    price_tree <- Binomial.tree(S, u, d, n)
    Sn <- price_tree[n + 1, ] # Stores the possible stock prices at the terminal steps.

    # Calculate the option value at the final step, i.e. maturity.
    if (type == "call") {
        fn <- sapply(Sn, function(x) {max(x - K, 0)})
    } else if (type == "put") {
        fn <- sapply(Sn, function(x) {max(K - x, 0)})
    } else {
        stop("Invalid option type. `type` must be specified to either 'call' or 'put'.")
    }

    # Calculate the option value at current time.
    price <- Binomial.rnv(fn, p, q, r, n, dt, K, S = price_tree, type = type, style = style)

    # Return output
    if (all) {
        list(
            "price" = price,
            "p" = p,
            "price_tree" = price_tree
        )
    } else {
        price
    }
}
\end{Rminted}

\newpage