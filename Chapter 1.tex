\chapter{Introduction} \label{cpt:Introduction}

\section{Report Structure}

Here in Chapter \ref{cpt:Introduction}, the literature review in Section \ref{sec:Literature Review} will introduce necessary definitions of financial options and option pricing methods that are included in \mintinline{R}|rmcop| package. This is accompanied by some useful formula deductions that will be seen useful in Chapter \ref{cpt:Pkg Dev}, where we will discuss the programming implementation of these models.

In Chapter \ref{cpt:Existing Packages}, we will introduce three existing option pricing packages in R (\mintinline{R}|derivmkts|, \mintinline{R}|fOptions|, \mintinline{R}|RQuantLib|), and explain their functionalities. We will also discuss their advantages and limitations.

In Chapter \ref{cpt:Pkg Dev}, we will discuss the package structure of \mintinline{R}|rmcop|. This includes: 1. a discussion on R's generic Object Oriented Programming (OOP) environment; 2. an outline of the package's structure; and 3. a detailed elaboration of the R implementation techniques of the financial models introduced in Section \ref{sec:Literature Review}.

Finally, in Chapter \ref{cpt:Results}, we will demonstrate examples and interpretations using \mintinline{R}|rmcop|. The report is then concluded by a discussion on \mintinline{R}|rmcop|'s limitations and plans for further development.

\section{Package Description} \label{sec:Pkg Description}

The R package \mintinline{R}|rmcop| is developed for pricing financial options. The name ``rmcop'' stands for ``R Monte Carlo Option Pricing''. The package supports Monte Carlo-based pricing for European vanilla options and European exotic options, including Asian, Barrier, Binary, and Lookback options. It also provides some deterministic methods for both European and American vanilla options pricing, including option pricing via the Binomial lattice tree and Black-Scholes formula.

Unlike most existing packages in the R Ecosystem, \mintinline{R}|rmcop| is developed through an object-oriented approach. User's argument inputs for pricing functions are encapsulated in corresponding objects, and pricing functions are themselves methods. This introduces ordered arguments control and easier functions application.

\section{Literature Review} \label{sec:Literature Review}

\subsection{Pricing of Financial Options}

An option is a common financial derivative\footnote{Security whose value depends on the value of an underlying (i.e. the related) asset.} in the market. Options can be roughly classified into two types, ``call'' or ``put''. A call option, ``gives its holder the right (but not the obligation) to purchase from the issuer a prescribed asset\footnote{Underlying asset/stock, the word "asset" and "stock" may be used interchangeably through this report.} for a prescribed price (strike price) at a prescribed time (maturity/expiration date) in the future.\cite{Higham2004}'' Similarly, a put option gives the right to sell an asset at the strike price at maturity.

There are two most common styles of financial options \cite{Higham2004}. A European option can only be exercised (i.e. executed) at maturity, whereas an American option can be exercised at any time prior to maturity (which is a more complex setting). Generally, an American option gives more freedom of exercising, so it is usually priced higher than a European option, giving all other parameters being equal.

Here is a hypothetical example. Suppose you hold a European call option on Apple's stock with a strike price of 150 and a maturity of 1 year. This option gives you the right to purchase a share of Apple's stock at exactly 150 on the day one year later. Let's assume that after a year, the market price of a share of Apple's stock is 160, and with the option in hand you, can purchase the stock with $160-150=10$ less. The result of £10 would be the benefit, or the \textbf{payoff}, that the option provides you. On the contrary, if after a year, the stock price is 140, which is less than the strike price of 150, there is no point exercising the option and purchasing the stock at a higher price. So the payoff of the option, in this case, would be zero.

The most typical "family" of options are the so-called ``vanilla options'' \cite{Higham2004}, which include no special or unusual features. For vanilla cases, a European call option has a payoff of $(S_T-K)^+ \coloneqq \max{(S_T-K,0)}$, and a European put option has a payoff of $(K-S_T)^+ \coloneqq \max{(K-S_T,0)}$ \cite{Glasserman2003} at the point of maturity $T$ in time. Here, $S_t$ is the underlying asset price at some future time $t\in[0,T]$, and $K$ is the option's strike price.

Options with any ``special or unusual features'' are called ``exotic options''. The exotic options whose pricing is supported by \mintinline{R}|rmcop| package is introduced below based on the definitions given in \textit{An Introduction to Financial Option Valuation} \cite{Higham2004}.

\begin{itemize} \label{lst:exotic_options}
    \item \textbf{Asian Options} Asian options' payoff are determined by the average price of the underlying asset throughout its life span, denoted $\bar{S}$.
    \begin{itemize}
		\item An average price Asian call option has payoff at the expiry $T$ given by $\max(\bar{S} - K)$.
		\item An average price Asian put option has payoff at the expiry $T$ given by $\max(K - \bar{S})$.
		\item An average strike Asian call option has payoff at the expiry $T$ given by $\max(S_T - \bar{S})$.
		\item An average strike Asian put option has payoff at the expiry $T$ given by $\max(\bar{S} - S_T)$.
	\end{itemize}
    \item \textbf{Barrier Options} Barrier options have a payoff that switches on or off depending on whether the asset crosses a pre-defined level (barrier) $B$.
    \begin{itemize}
		\item A down-and-out (knock-out) call option has a payoff that is zero if the asset crosses some predefined barrier $B<S_0$ at some time in $[0, T]$. If the barrier is not crossed then the payoff becomes that of a European call, $\max(S_T - K, 0)$.
		\item A down-and-in (knock-in) call option has a payoff that is zero unless the asset crosses some predefined barrier $B<S_0$ at some time in $[0, T]$. If the barrier is crossed then the payoff becomes that of a European call, $\max(S(T) - K, 0)$.
    \end{itemize}
    \item \textbf{Binary Options} A binary (a.k.a. cash-or-nothing) option have payoff at expiry being either some specified value $A$ or zero.
    \begin{itemize}
		\item A binary call option has payoff $A$ if $S_T>K$ and zero otherwise.
		\item A binary put option has payoff $A$ if $S_T<K$ and zero otherwise.
	\end{itemize}
    \item \textbf{Lookback Options} The payoff for a lookback option depends upon either the maximum $S^{\max}$ or the minimum value $S^{\min}$ attained by the asset throughout the price trajectory.
    \begin{itemize}
		\item A fixed strike lookback call option has payoff at expiry $T$ given by $\max(S^{\max}-K, 0)$.
		\item A fixed strike lookback put option has payoff at expiry $T$ given by $\max(K-S^{\min}, 0)$.
		\item A floating strike lookback call option has payoff at expiry $T$ given by $S_T -S^{\min}$.
		\item A floating strike lookback put option has payoff at expiry $T$ given by $S^{\max} -S_T$.
    \end{itemize}
\end{itemize}

\subsection{Brownian Motion}

A \textit{stochastic process}, by definition \cite{Iacus2008}, is a family of random variables $\{X_\gamma,\gamma\in\Gamma\}$ defined on $\Omega\times\Gamma$ taking values in $\mathbb{R}$. Thus, the random variables of the family (measurable for every $\gamma\in\Gamma$) are functions of the form:

\begin{align*}
X(\gamma,\omega):\Gamma\times\Omega\mapsto\mathbb{R}
\end{align*}

Where $(\omega,\mathcal{A},P)$ is a probability space described by samples $\omega$, action space $\mathcal{A}$, and probability measure $P$.

For $\Gamma=\mathbb{N}$, we have a \textit{discrete process}, and for $\Gamma\subset\mathbb{R}$, we have a \textit{continuous process}. In our context, we will consider $\Gamma$ to represent the scope in time and takes values from $0$ to maturity $T$.

A \textit{stochastic process} $\{X(t), t\geq 0\}$ is said to be a \textit{Brownian motion} if \cite{Iacus2008}:

\begin{enumerate}
	\item $X(0)=0$;
	\item $\{X(t),t\geq0\}$ has stationary and independent increments.
	\item for every $t>0$, $X(t)\sim N(0,\sigma^2 t)$.
\end{enumerate}

In simple terms, a Brownian motion starts at a point in time, and for each time step, we move to a new point which is determined by a distribution centered on the current location.

A \textit{Standard Brownian Motion} (i.e. Wiener Process, abbreviated as SBM) inherits the properties above, and has a volatility measure of zero (i.e. $\sigma=0$). So for an SBM $W$, we have $W(t) \sim N(0,t)$.

\begin{figure}[H]
	\centering
	\includegraphics[scale = 0.5]{wiener_process.jpeg}
	\caption{Trajectories of Wiener Process} \label{img:wiener_process}
\end{figure}

Figure \ref{img:wiener_process} demonstrates the trajectories of a one-dimensional SBM, where each grey line represents a simulated path of a Wiener process.

Notice that the expectation of an SBM at any time $t\geq0$ is zero, this makes an SBM a \textit{martingale}, which is a stochastic process that ``does not tend to rise or fall\footnote{That is, the expectation $E[W(t+k)]=W(t)$ for any increment in time $k$.} \cite{Shreve2004}''. If $S_t$ is the value of an SBM at time $t$, assume $S_0=0$, its SBM dynamics can be defined by the Stochastic Differential Equation (SDE):

\begin{align*}
dS_t = S_t dW_t
\end{align*}

We can generalise the scenarios by considering drifts through time (measured by $\mu$) and scaling the amount of diffusion with respect to time (measured by $\sigma$). For $S_t$ that follows a \textit{Geometric Brownian Motion} (GBM), its dynamics can be described by the SDE \cite{Shreve2004}:

\begin{align*}
dS_t = \mu S_t dt + \sigma S_t dW_t
\end{align*}

Where $\mu S_t dt$ is called the \textit{drift term} and $\sigma S_t dW_t$ the \textit{diffusion term}.

Expanding on $S_t$, for a function $f:\mathbb{R}\mapsto\mathbb{R}$, the GBM $f(S_t)$ is described as, according to Itô's rule \cite{Shreve2004}:

\begin{align*}
df(S_t) = \left[\mu_t\frac{\partial f}{\partial S}(S_t)+\frac{1}{2}\frac{\partial^2 f}{\partial S^2}(S_t)\right]dt + \sigma_t\frac{\partial f}{\partial S}dW_t
\end{align*}

Using the Itô-Doeblin rule for $f:\mathbb{R}\mapsto\mathbb{R}$ \cite{Shreve2004}, we can derive the following explicit solution to the SDE:

\begin{align} \label{eq:GBM_explicit}
f(S_t) = f(S_0) + \int_{0}^{t}{\mu_t\frac{\partial f}{\partial S}(S_t)}dt + \frac{1}{2}\int_{0}^{t}{\sigma_t^2\frac{\partial^2 f}{\partial S^2}(S_t)}dt + \int_{0}^{t}{\sigma_t\frac{\partial f}{\partial S}(S_t)}dW_t
\end{align}

Equation \ref{eq:GBM_explicit} will be shown useful in the Black-Scholes model introduced in the following section.

\subsection{Black-Scholes Model}

The first quantitative method in option pricing (perhaps in all finance) was developed by a French mathematician, Louis Bachelier, in 1900. In his work \textit{The Theory of Speculation} \cite{Bachelier1900}, he deducted deterministic formulas for pricing European (vanilla) calls and put options as follows:

\begin{align*}
C(S, T) &= SN(\frac{S - X}{\sigma\sqrt{t}}) - XN(\frac{S - X}{\sigma \sqrt{t}}) + \sigma\sqrt{t}N(\frac{S - X}{\sigma \sqrt{t}}) \\
P(S, T) &= XN(\frac{S - X}{\sigma \sqrt{t}}) - SN(\frac{S - X}{\sigma \sqrt{t}}) + \sigma\sqrt{t}N(\frac{S - X}{\sigma \sqrt{t}})
\end{align*}

Being the earliest approach, Bachelier's formula had already outlined the relationship between the option price and asset price $S$, strike price $X$, and volatility measure $\sigma$, which are essential fragments in modern formulas. However, based on limited data, Bachelier's solution was built under some unrealistic assumptions \cite{Bachelier1900}. The normality assumption violates the non-negativity of the stock price, and the formula's discrete measure in time omitted the effect of continuous movements in the stock price. Also, the formula did not discount the effect of interest rates. These errors cause Bachelier's model fails to price options accurately.

It wasn't until the 1960s had further improvements been made to quantitative option pricing. In 1961, Case Sprenkle \cite{Sprenkle1961} introduced the Sprenkle formula. The formula addressed the above issues by describing the stock price by the more suitable log-normal distribution\footnote{An log-normal distribution is defined on $(0,\infty)$, which addressed the non-negativity issue.} and discounting for the effect of interest rate, which successfully explained the time value of an option. In the following decade, improvements have been made by scholars such as Boness and Samuelson \cite{BS1973}, who introduced empirical constants to increase the effectiveness of Sprenkle's model.

The model was finalised by Black and Scholes in 1973 \cite{BS1973}, who explained the stock price movement by GBM. The GBM was described in the form of a Stochastic Differential Equation (SDE), which effectively models the continuity of price movement. The solution (derived through Itô's lemma) of the Black-Scholes formula under a risk-neutral approach\footnote{The risk-neutral approach, in simple terms, is constructing a portfolio at a moment in time such that the portfolio value will be identical at the next moment in time regardless of the price movement, so the portfolio will be riskless to the price movement.} eliminates the empirical measure of Sprenkle et al's model, which resulted in an objective and deterministic estimation of the option price, as is used by most contemporary pricing methods.

Recall the explicit form of an SDE $f(S_t)$ given by the Itô-Doeblin rule in Equation \ref{eq:GBM_explicit}. We assign $f(S_t)=\log(S_t)$\footnote{The transformation constructing the log-normal distribution.} where $\log$ is the natural logarithm. The solution is:

\begin{align} \label{eq:BS_log_explicit}
\log(S_t) = \log(S_t) + (\mu - \frac{\sigma^2}{2})t + \sigma W_t
\end{align}

Because $W_t\sim N(0, t)$, we can see that Equation \ref{eq:BS_log_explicit} shows that:

\begin{align*}
\log(S_t)\sim N(\log(S_0)+(\mu - \frac{\sigma^2}{2})t, \sigma^2 t)
\end{align*}

This implies that $S_t$ follows a \textit{log-normal} distribution. A log-normal distribution takes only a positive value, which addressed the negativity issue caused by Gaussian stock price models, like the one developed by Bachelier \cite{Bachelier1900}.

By taking the exponential on the two sides of the equation, one can obtain the following expression:

\begin{align} \label{eq:BS_explicit}
S_t = S_0\exp{\left\{(\mu-\frac{\sigma^2}{2})t + \sigma W_t\right\}}
\end{align}

Which describes the stock price $S_t$ at anytime $t\in[0,T]$. When simulating real market conditions, one may replace the factor $\mu$ with $r$ and $q$, which stands for the (fixed) interest rate and dividend yield rate, respectively.

\begin{align*}
S_t = S_0\exp{\left\{(r-q-\frac{\sigma^2}{2})t + \sigma W_t\right\}}
\end{align*}

Under the risk-neutral assumption, the payoff of a (European vanilla) call option with strike price $K$ and expiration $T$ is given by the expectation $E[e^{-rT}(S(T)-K)^+]$, and the corresponding payoff of a (European vanilla) put option at the same strike price and expiration is given by $E[e^{-rT}(K-S(T))^+]$. Using the solution of the Black-Scholes SDE we can deterministically evaluate the payoffs as follows \cite{Higham2004}:

\begin{align*}
&C(S(0), T) = S(0)\Phi(d_1) - e^{-rT}K\Phi(d_2) \\ 
&P(S(0), T) = e^{-rT}K\Phi(-d_2) - S(0)\Phi(-d_1)
\end{align*}

Where $\Phi$ is the cumulative normal distribution, $d_1 \coloneqq \frac{\log(S(0)/K)+(r+\frac{1}{2}\sigma^2)T}{\sigma\sqrt{T}}$, and $d_2 = d_1 - \sigma\sqrt{T}$. The Equations \ref{eq:BS Formula Call} and \ref{eq:BS Formula Put} are the so-called Black-Scholes formula for European options' price.

A generalised solution which addressed the impact of the presence of dividends (assuming the option of interest has an annual dividend yield rate of $\delta$) specified as \cite{Glasserman2003}:

\begin{align}
	&C(S(0), T) = e^{-\delta t}S(0)\Phi(d_1) - e^{-rT}K\Phi(d_2) \label{eq:BS Formula Call} \\
	&P(S(0), T) = e^{-rT}K\Phi(-d_2) - e^{-\delta t}S(0)S(0)\Phi(-d_1) \label{eq:BS Formula Put}
\end{align}

Where now $d_1 \coloneqq \frac{\log(S(0)/K)+(r-\delta+\frac{1}{2}\sigma^2)T}{\sigma\sqrt{T}}$, and $d_2$ has the same relationship described above.

A further result of the American option cases is such that an American call option is never optimal (i.e. the estimated value would be the same as the European call option under the same conditions); and that an American put option's value does not have an analytical form and required numerical methods to calculate\footnote{This is related to the studies on Monte Carlo pricing for American Options, which is beyond the current scope of \mintinline{R}|rmcop|.}.

\subsection{Binomial Lattice Model}

In 1979, based on the risk-neutral approach used by the Black-Scholes model, Cox, Ross and Rubinstein (CRR) \cite{CRR1979} introduced a Binomial lattice tree model for modelling stock prices.

To construct a Binomial tree of stock prices, one breaks down the option's life $[0,T]$ into $n$ time steps with fixed interval $\Delta t=T/n$. For each time step, the stock price can move either up by a factor $u$ with probability $\hat{p}$ or down by a factor $d$ with probability $\hat{q}=1-\hat{p}$.

\begin{figure}[H]
    \centering
    \[\tikz{
        \node (a) at (0, 2) {$S$};
        \node (b) at (2.5, 3) {$uS$};
        \node (c) at (2.5, 1) {$dS$};
		\node (d) at (5, 4) {$u^2S$};
        \node (e) at (5, 2) {$udS$};
		\node (f) at (5, 0) {$d^2S$};
        \path[->] (a) edge node [above left] {$\hat{p}$} (b)
				  (a) edge node [below left] {$1-\hat{p}$} (c);
		\path[->] (b) edge node [above left] {$\hat{p}$} (d)
        		  (b) edge node [above right] {$1-\hat{p}$} (e);
		\path[->] (c) edge node [below right] {$\hat{p}$} (e)
        		  (c) edge node [below left] {$1-\hat{p}$} (f);
    }\]
    \caption{Binomial Lattice Tree for Stock Price with 2 Steps} \label{gph:binomial_tree}
\end{figure}

As Figure \ref{gph:binomial_tree} shows, we will obtain a total of $n+1$ nodes at maturity $T$. For each node at the final step, we can obtain the corresponding payoff of an option related to that stock price. For example, recall from the above section, the $n+1$ possible payoffs $P(S(T))$ of an vanilla European call option would be:

\begin{align}
P(S(T))=(S(T)-K)^+
\end{align}

The probability $\hat{p}$ is set to be ``risk neutral'' in a way such that:

\begin{align*}
S(t_{i}) &= e^{-r\Delta t}\mathbb{E}[S(t_{i+1})] \\
		 &= e^{-r\Delta t}[\hat{p}uS(t_i)+\hat{q}dS(t_i)],\quad\text{for }i=0,...,n-1
\end{align*}

And it can be calculated using the formula \cite{CRR1979}:

\begin{align} \label{eq:binomial_riskless_p}
\hat{p} = \frac{e^{r\Delta t}-d}{u-d}
\end{align}

This means the current node's stock price is equal to the expected value of the two future nodes it can go in the next time step, discounting the interest. By this risk-neutral setup, a portfolio which consists only of this stock would have a fixed return independent from price movement.

Under this setting, the option payoff at each node is computed in a similar way. Suppose payoff $P_{ij}$ denotes the payoff of the $j$th node at time step $i$, it can be computed via recursive form:

\begin{align} \label{eq:binomial_recursive}
P_{ij} &= e^{-rT}[\hat{p}P_{(i+1),j+1} + \hat{q}P_{(i+1),j}]
\end{align}

As mentioned above, the payoff of an option at maturity $T$ is obvious to obtain. By applying this recursive method forward from $t=T$ back to $t=0$, we will obtain a deterministic estimate of the current payoff (fair value) of the option of interest.

\subsection{Monte Carlo Option Pricing}

As the financial market develops and more complicated options emerge, in many realistic cases, one cannot find a deterministic solution for pricing options. However, thanks to the advancement of computer power, one can simulate price trajectories for enormous times, and estimate the payoff of the option of interest by simply taking the average of the option payoff under each simulated trajectory. This method is known as the Monte Carlo method. The very first attempt at applying a computational method in option pricing is by Phelim Boyle in 1977. A more serious (and effective) approach was introduced by Paul Glasserman in the 1990s based on the Black-Scholes model. Until now, the field of Monte Carlo option pricing is still under active development and is widely used by "quants". The contents below will elaborate implementation method as the one introduced in Glasserman's \textit{Monte Carlo Methods in Financial Engineering} \cite{Glasserman2003}.

We know that a GBM is a continuous process, and its dynamics are described by an SDE with respect to the $t$ through infinitesimal $dt$. In order to simulate the process, we need to discretise $dt$ to the computable $\Delta t$.

Recall the explicit form of the GBM (Equation \ref{eq:BS_explicit}) introduced in the Black-Scholes model, we can expand on it to describe the change in asset price $S_{t+\Delta t}$ as:

\begin{align*}
S_{t+\Delta t} = S_t\exp{\left\{(\mu-\frac{\sigma^2}{2})\Delta t+\sigma W_{\Delta t}\right\}}
\end{align*}

Knowing that $W_t\sim N(0,t)$, suppose we have $Z\colonsim N(0,1)$, we can discretise $W_t=Z\sqrt{t}$. So we can modify the above equation to:

\begin{align*}
S_{t+\Delta t} = S_t\exp{\left\{(\mu-\frac{\sigma^2}{2})\Delta t+\sigma Z \sqrt{\Delta t}\right\}}
\end{align*}

Suppose we discretise the continuous time interval $[0,T]$ into $m$ time steps with fixed length $\Delta t=\frac{T}{m}$. We can derive the iterative formula:

\begin{align} \label{eq:mc_explicit}
S_{i} = S_{i-1}\exp{\left\{(\mu-\frac{\sigma^2}{2})\Delta t+\sigma Z_i \sqrt{\Delta t}\right\}},\qquad i=1,...,m
\end{align}

We can also modify the equation to address the effect of interest rate and dividend yield rate according to what's mentioned below Equation \ref{eq:BS_explicit}:

\begin{align} \label{eq:mc_explicit_rq}
S_{i} = S_{i-1}\exp{\left\{(r-q-\frac{\sigma^2}{2})\Delta t+\sigma Z_i \sqrt{\Delta t}\right\}},\qquad i=1,...,m
\end{align}

By simulating $m$ standard normal random variables $Z_1,...,Z_m$, we will be able to generate a full trajectory of the price movements. We will represent this using the below pseudocode \cite{Glasserman2003}.

\begin{algorithmic}[H]
\For{$i \gets 1,...,m$}
	\State $Z_{i} \gets N(0,1)$
	\State $S_{i} \gets S_{i-1}\exp{\left\{(\mu-\frac{\sigma^2}{2})\Delta t+\sigma Z_i \sqrt{\Delta t}\right\}}$
\EndFor
\end{algorithmic}

Noticing that as $W_T\sim N(0,T)$, and that individual increments for Brownian motions are independent, the approximation of $W_T$ using $Z_i$'s is lossless (i.e. the number of time steps/length of $\Delta t$, has no effect on how well the approximation will be). For path-independent options\footnote{Options whose payoff only depends on the stock price at maturity, $S_T$.}, it may be wise to take $m=1$, i.e. $\Delta t=T$ to minimise the computational cost.

If we are to simulate $n$ times, for the $i$th trajectory, $i=1,...,n$, one may employ formulas introduced in List \ref{lst:exotic_options} and discounting the effect of interest rate and dividend rate to obtain the corresponding discounted payoff (i.e. fair price) of the option, $C_i$. For example, the discounted payoff of a Vanilla call option, at maturity, can be calculated as:

\begin{align} \label{eq:discount_vanilla_payoff}
C_i = e^{-rT}(S_T-K)^+
\end{align}

The Monte Carlo estimator for the option price is then simply the arithmetic mean $\hat{C}_n\coloneqq\frac{1}{n}\sum_{i=1}^{n}{C_i}$. This estimator is \textit{unbiased} and \textit{strongly consistent} \cite{Glasserman2003}.

For large enough $n$, we can construct a confidence interval using the standard error, given by \cite{Glasserman2003}:

\begin{align} \label{eq:mc_SE}
SE_C &= \frac{s_C}{\sqrt{n}} \\
 s_C &= \sqrt{\frac{1}{n-1}\sum_{i=1}^{n}{(C_i-\hat{C}_n)^2}}
\end{align}

Glasserman in his work \cite{Glasserman2003} also indicated the criterion for the simulation estimators' efficiency: computing time, bias, and variance. The estimators considered in our package are unbiased, and variance reduction techniques are beyond our scope. So we will only discuss some R programming methods in reducing the computing time to increase estimation efficiency, as we will see in Chapter \ref{cpt:Pkg Dev}.

\newpage