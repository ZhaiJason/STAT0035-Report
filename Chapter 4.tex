\chapter{Package Development}

\section{Package Structure}

\subsection{Objective Oriented Programming in R}

Existing packages in R ecosystem, such as \mintinline{R}|fOptions|, \mintinline{R}|derivmkts|, and \mintinline{R}|RQuantLib| provides comprehensive pricing algorithms for financial derivatives. However, their implementation of models are procedural oriented.



Generic S3 method.


\section{Deterministic Methods}

\subsection{Black-Scholes}
The Black-Scholes model perceive the movement of the stock price as an Geometric Brownian motion, explained by the stochastic differential equation:

\begin{align*}
    \frac{dS(t)}{S(t)} = \mu (S(t), t)dt + \sigma dW(t)
\end{align*}


\subsection{Binomial Lattice Tree}


Introduced by Cox, Ross, and Rubinstein in 1979, the Binomial Model 

The Cox, Ross, Rubinstein (CRR) Binomial Model was introduced by 


\subsection{Trinomial Lattice}
 
Extending the Binomial model, the Trinomial Lattice model was introduced by Phelim Boyle in 1988 \cite{Boyle1988}.

\begin{Rminted}
Trinomial <- function(K, S, u, r, t, n, sigma = 0,
    type = "call",
    style = "European",
    all = FALSE,
    plot = FALSE) {

}
\end{Rminted}

\subsection{Discussion on Multinomial Option Pricing Models \& Their Relationship with Binomial Model}

\newpage

\section{Monte Carlo Methods}

\subsection{Vanilla Option Pricing}

\subsection{Asian Option Pricing}

Different from a vanilla option whose payoff only depends on stock price at maturity (i.e. $t = T$), an Asian option's payoff is determined by the mean stock price throughout the option's life $\bar{S}$ and the strike price $K$. To calculate $\bar{S}$, we need to consider the price movement throughout $t \in [0, T]$, making the pricing of an Asian option path-dependent.

% Cite from Glasserman

A generalised case of the stock pricing model is given by:

\begin{align}
    dS(t) = rS(t)dt + \sigma(S(t))S(t)dW(t)
\end{align}

By discritise the infinitesimal $dt$ to $\Delta t$ using Euler approximation, we can obtain the form:

\begin{align}
    S(t + \Delta t) = S(t) + rS(t)\Delta t + \sigma(S(t))S(t)\sqrt{\Delta t}Z
\end{align}

Where $Z \sim N(0,1)$. Taking the logrithm of the above expression, we have:

\begin{align}
    log(S(t + \Delta t)) = log(S(t))
\end{align}

\newpage