\chapter{Existing Option Pricing Packages within R Ecosystem} \label{cpt:Existing Packages}

\section{Introduction}

In this chapter, we will introduce three R packages on CRAN which are used for pricing financial options. We will comment on each of their strengths and weaknesses.

\section{Packages Review}

\subsection{derivmkts} \label{sub:derivmkts}

The R package \mintinline{R}|derivmkts| \cite{McDonald2022}, which abbreviates for ``derivative markets'', provides a collection of functions for pricing and analysing financial derivatives. It includes options pricing methods, including Monte Carlo simulations and binomial tree method, as well as other market analysis tools including calculating Greeks, implied volatility, and so on.

The package \mintinline{R}|derivmkts|'s functions are functional-specific and relatively easy to use. For instance, its \mintinline{R}|asianmc| functions provide pricing calculations exclusively for Asian options using the Monte Carlo method, and \mintinline{R}|perpetual| functions are specialised for pricing perpetual options\footnote{Options with maturity $T=\infty$, i.e. an option that never matures.} only. Implementing these functions is simple, but this high function independence lacks expandability. Without reusing existing functions, new functions have to be created bottom-up. As a result, the package only provides limited functions.

On the other hand, the package document is sparse and sometimes incomplete. Algorithms such as the \mintinline{R}|binom| for generating the Binomial tree and \mintinline{R}|simprice| for running Monte Carlo price simulations have not been benchmarked against other sources and are computationally inefficient. For example, the below codes benchmarked price trajectories sampling using \mintinline{R}|simprice| from \mintinline{R}|derivmkts| against the function in \mintinline{R}|rmcop| using the package \mintinline{R}|microbenchmark| \cite{Mersmann2022}.

\begin{Rminted}
func.rmcop <- function() { # Monte Carlo pricie simulation using rmcop
    op <- option("european", type = "call", K = 40, t = 0.25)
    op.env <- option.env(S = 40, r = 0.08, q = 0, sigma = 0.3, n = 100, steps = 100)
    price.option(op, op.env)
}

func.derivmkts <- function() { # Monte Carlo pricie simulation using derivmkts
    simprice(s0 = 40, v = 0.3, r = 0.08, tt = 0.25, d = 0,  trials = 100, periods = 100, jump = FALSE)
}

microbenchmark::microbenchmark(
    func.rmcop(),
    func.derivmkts()
)
\end{Rminted}

The results are:

\begin{Rminted}
> Unit: microseconds
>             expr     min      lq      mean  median       uq     max neval cld
>     func.rmcop()   859.9   951.0  1040.669   985.4  1055.45  2985.8   100  a 
> func.derivmkts() 28316.3 28691.1 31946.188 29055.6 29936.10 74694.6   100  b
\end{Rminted}

From which we can see that the average computing time of simulating 100 trajectories with 100 times steps on each trajectory is over 30 times longer for \mintinline{R}|derivmkts| than \mintinline{R}|rmcop|.

The reason is because \mintinline{R}|derivmkts|, as a supplement programme for educational purposes, had its functions coded in a very direct way corresponding to the underlying formulae. This results in a pervasive use of for-loops in R without any specific adjustments to improve the computation speed.

% TODO: left out here

\subsection{fOptions} \label{sub:fOptions}

Similar to \mintinline{R}|derivmkts|, \mintinline{R}|fOptions| \cite{Wuertz2017} is also a supplement package for educational purpose. It provides a collection of functions for pricing and analyzing various financial options, such as European, American, Asian, exotic, and barrier options. The package also includes methods for calculating implied volatility, greeks, and binomial trees. It is useful for researchers and practitioners who want to apply option pricing models to real-world data and scenarios.

In contrast to \mintinline{R}|derivmkts|, it has a clearer documentation and examples. However, the package functions still lacks computing efficiency (having the same issue as the \mintinline{R}|derivmkts|). A more crucial issue with the package is that it is no longer supported by the current R environment, and its development was terminated since 2020.

\subsection{RQuantLib} \label{sub:RQuantLib}

Both \mintinline{R}|derivmkts| and \mintinline{R}|fOptions| are supplementary materials for educational purpose. \mintinline{R}|RQuantLib| \cite{Eddel2022}, on the other hand, is a professional tool for industrial computational finance.

\mintinline{R}|RQuantLib| is an R interface to the QuantLib software, an open-source library for quantitative finance based on C++. The purpose of RQuantLib is to enable R users to perform complex financial calculations and simulations using the functionality of QuantLib. Some of the strengths of RQuantLib are:

- It covers a wide range of financial instruments and models, such as bonds, options, swaps, interest rate models, volatility models, etc.
- It allows for easy integration with other R packages and data sources, such as time series analysis, graphics, optimization, etc.
- It offers high performance and reliability due to the use of C++ code and rigorous testing.

However, due to its highly professional nature, it requires pre-installation and configuration of QuantLib software on the system, which may be challenging for many users. Also, because it is a supplement of the QuantLib library, it's own documentation and examples are very limited compared to other R packages, making it not a user-friendly choice for less-intensive option pricing purpose.

\newpage